%% bare_jrnl.tex
%% V1.3
%% 2007/01/11
%% by Michael Shell
%% see http://www.michaelshell.org/
%% for current contact information.
%%
%% This is a skeleton file demonstrating the use of IEEEtran.cls
%% (requires IEEEtran.cls version 1.7 or later) with an IEEE journal paper.
%%
%% Support sites:
%% http://www.michaelshell.org/tex/ieeetran/
%% http://www.ctan.org/tex-archive/macros/latex/contrib/IEEEtran/
%% and
%% http://www.ieee.org/


\documentclass[journal]{IEEEtran}
\usepackage{blindtext}
\usepackage{graphicx, amsmath, amssymb}

% Some very useful LaTeX packages include:
% (uncomment the ones you want to load)


% *** MISC UTILITY PACKAGES ***
%
%\usepackage{ifpdf}
% Heiko Oberdiek's ifpdf.sty is very useful if you need conditional
% compilation based on whether the output is pdf or dvi.
% usage:
% \ifpdf
%   % pdf code
% \else
%   % dvi code
% \fi
% The latest version of ifpdf.sty can be obtained from:
% http://www.ctan.org/tex-archive/macros/latex/contrib/oberdiek/
% Also, note that IEEEtran.cls V1.7 and later provides a builtin
% \ifCLASSINFOpdf conditional that works the same way.
% When switching from latex to pdflatex and vice-versa, the compiler may
% have to be run twice to clear warning/error messages.






% *** CITATION PACKAGES ***
%
%\usepackage{cite}
% cite.sty was written by Donald Arseneau
% V1.6 and later of IEEEtran pre-defines the format of the cite.sty package
% \cite{} output to follow that of IEEE. Loading the cite package will
% result in citation numbers being automatically sorted and properly
% "compressed/ranged". e.g., [1], [9], [2], [7], [5], [6] without using
% cite.sty will become [1], [2], [5]--[7], [9] using cite.sty. cite.sty's
% \cite will automatically add leading space, if needed. Use cite.sty's
% noadjust option (cite.sty V3.8 and later) if you want to turn this off.
% cite.sty is already installed on most LaTeX systems. Be sure and use
% version 4.0 (2003-05-27) and later if using hyperref.sty. cite.sty does
% not currently provide for hyperlinked citations.
% The latest version can be obtained at:
% http://www.ctan.org/tex-archive/macros/latex/contrib/cite/
% The documentation is contained in the cite.sty file itself.






% *** GRAPHICS RELATED PACKAGES ***
%
\ifCLASSINFOpdf
  % \usepackage[pdftex]{graphicx}
  % declare the path(s) where your graphic files are
  % \graphicspath{{../pdf/}{../jpeg/}}
  % and their extensions so you won't have to specify these with
  % every instance of \includegraphics
  % \DeclareGraphicsExtensions{.pdf,.jpeg,.png}
\else
  % or other class option (dvipsone, dvipdf, if not using dvips). graphicx
  % will default to the driver specified in the system graphics.cfg if no
  % driver is specified.
  % \usepackage[dvips]{graphicx}
  % declare the path(s) where your graphic files are
  % \graphicspath{{../eps/}}
  % and their extensions so you won't have to specify these with
  % every instance of \includegraphics
  % \DeclareGraphicsExtensions{.eps}
\fi
% graphicx was written by David Carlisle and Sebastian Rahtz. It is
% required if you want graphics, photos, etc. graphicx.sty is already
% installed on most LaTeX systems. The latest version and documentation can
% be obtained at:
% http://www.ctan.org/tex-archive/macros/latex/required/graphics/
% Another good source of documentation is "Using Imported Graphics in
% LaTeX2e" by Keith Reckdahl which can be found as epslatex.ps or
% epslatex.pdf at: http://www.ctan.org/tex-archive/info/
%
% latex, and pdflatex in dvi mode, support graphics in encapsulated
% postscript (.eps) format. pdflatex in pdf mode supports graphics
% in .pdf, .jpeg, .png and .mps (metapost) formats. Users should ensure
% that all non-photo figures use a vector format (.eps, .pdf, .mps) and
% not a bitmapped formats (.jpeg, .png). IEEE frowns on bitmapped formats
% which can result in "jaggedy"/blurry rendering of lines and letters as
% well as large increases in file sizes.
%
% You can find documentation about the pdfTeX application at:
% http://www.tug.org/applications/pdftex





% *** MATH PACKAGES ***
%
%\usepackage[cmex10]{amsmath}
% A popular package from the American Mathematical Society that provides
% many useful and powerful commands for dealing with mathematics. If using
% it, be sure to load this package with the cmex10 option to ensure that
% only type 1 fonts will utilized at all point sizes. Without this option,
% it is possible that some math symbols, particularly those within
% footnotes, will be rendered in bitmap form which will result in a
% document that can not be IEEE Xplore compliant!
%
% Also, note that the amsmath package sets \interdisplaylinepenalty to 10000
% thus preventing page breaks from occurring within multiline equations. Use:
%\interdisplaylinepenalty=2500
% after loading amsmath to restore such page breaks as IEEEtran.cls normally
% does. amsmath.sty is already installed on most LaTeX systems. The latest
% version and documentation can be obtained at:
% http://www.ctan.org/tex-archive/macros/latex/required/amslatex/math/





% *** SPECIALIZED LIST PACKAGES ***
%
%\usepackage{algorithmic}
% algorithmic.sty was written by Peter Williams and Rogerio Brito.
% This package provides an algorithmic environment fo describing algorithms.
% You can use the algorithmic environment in-text or within a figure
% environment to provide for a floating algorithm. Do NOT use the algorithm
% floating environment provided by algorithm.sty (by the same authors) or
% algorithm2e.sty (by Christophe Fiorio) as IEEE does not use dedicated
% algorithm float types and packages that provide these will not provide
% correct IEEE style captions. The latest version and documentation of
% algorithmic.sty can be obtained at:
% http://www.ctan.org/tex-archive/macros/latex/contrib/algorithms/
% There is also a support site at:
% http://algorithms.berlios.de/index.html
% Also of interest may be the (relatively newer and more customizable)
% algorithmicx.sty package by Szasz Janos:
% http://www.ctan.org/tex-archive/macros/latex/contrib/algorithmicx/




% *** ALIGNMENT PACKAGES ***
%
%\usepackage{array}
% Frank Mittelbach's and David Carlisle's array.sty patches and improves
% the standard LaTeX2e array and tabular environments to provide better
% appearance and additional user controls. As the default LaTeX2e table
% generation code is lacking to the point of almost being broken with
% respect to the quality of the end results, all users are strongly
% advised to use an enhanced (at the very least that provided by array.sty)
% set of table tools. array.sty is already installed on most systems. The
% latest version and documentation can be obtained at:
% http://www.ctan.org/tex-archive/macros/latex/required/tools/


%\usepackage{mdwmath}
%\usepackage{mdwtab}
% Also highly recommended is Mark Wooding's extremely powerful MDW tools,
% especially mdwmath.sty and mdwtab.sty which are used to format equations
% and tables, respectively. The MDWtools set is already installed on most
% LaTeX systems. The lastest version and documentation is available at:
% http://www.ctan.org/tex-archive/macros/latex/contrib/mdwtools/


% IEEEtran contains the IEEEeqnarray family of commands that can be used to
% generate multiline equations as well as matrices, tables, etc., of high
% quality.


%\usepackage{eqparbox}
% Also of notable interest is Scott Pakin's eqparbox package for creating
% (automatically sized) equal width boxes - aka "natural width parboxes".
% Available at:
% http://www.ctan.org/tex-archive/macros/latex/contrib/eqparbox/





% *** SUBFIGURE PACKAGES ***
%\usepackage[tight,footnotesize]{subfigure}
% subfigure.sty was written by Steven Douglas Cochran. This package makes it
% easy to put subfigures in your figures. e.g., "Figure 1a and 1b". For IEEE
% work, it is a good idea to load it with the tight package option to reduce
% the amount of white space around the subfigures. subfigure.sty is already
% installed on most LaTeX systems. The latest version and documentation can
% be obtained at:
% http://www.ctan.org/tex-archive/obsolete/macros/latex/contrib/subfigure/
% subfigure.sty has been superceeded by subfig.sty.



%\usepackage[caption=false]{caption}
%\usepackage[font=footnotesize]{subfig}
% subfig.sty, also written by Steven Douglas Cochran, is the modern
% replacement for subfigure.sty. However, subfig.sty requires and
% automatically loads Axel Sommerfeldt's caption.sty which will override
% IEEEtran.cls handling of captions and this will result in nonIEEE style
% figure/table captions. To prevent this problem, be sure and preload
% caption.sty with its "caption=false" package option. This is will preserve
% IEEEtran.cls handing of captions. Version 1.3 (2005/06/28) and later
% (recommended due to many improvements over 1.2) of subfig.sty supports
% the caption=false option directly:
%\usepackage[caption=false,font=footnotesize]{subfig}
%
% The latest version and documentation can be obtained at:
% http://www.ctan.org/tex-archive/macros/latex/contrib/subfig/
% The latest version and documentation of caption.sty can be obtained at:
% http://www.ctan.org/tex-archive/macros/latex/contrib/caption/




% *** FLOAT PACKAGES ***
%
%\usepackage{fixltx2e}
% fixltx2e, the successor to the earlier fix2col.sty, was written by
% Frank Mittelbach and David Carlisle. This package corrects a few problems
% in the LaTeX2e kernel, the most notable of which is that in current
% LaTeX2e releases, the ordering of single and double column floats is not
% guaranteed to be preserved. Thus, an unpatched LaTeX2e can allow a
% single column figure to be placed prior to an earlier double column
% figure. The latest version and documentation can be found at:
% http://www.ctan.org/tex-archive/macros/latex/base/



%\usepackage{stfloats}
% stfloats.sty was written by Sigitas Tolusis. This package gives LaTeX2e
% the ability to do double column floats at the bottom of the page as well
% as the top. (e.g., "\begin{figure*}[!b]" is not normally possible in
% LaTeX2e). It also provides a command:
%\fnbelowfloat
% to enable the placement of footnotes below bottom floats (the standard
% LaTeX2e kernel puts them above bottom floats). This is an invasive package
% which rewrites many portions of the LaTeX2e float routines. It may not work
% with other packages that modify the LaTeX2e float routines. The latest
% version and documentation can be obtained at:
% http://www.ctan.org/tex-archive/macros/latex/contrib/sttools/
% Documentation is contained in the stfloats.sty comments as well as in the
% presfull.pdf file. Do not use the stfloats baselinefloat ability as IEEE
% does not allow \baselineskip to stretch. Authors submitting work to the
% IEEE should note that IEEE rarely uses double column equations and
% that authors should try to avoid such use. Do not be tempted to use the
% cuted.sty or midfloat.sty packages (also by Sigitas Tolusis) as IEEE does
% not format its papers in such ways.


%\ifCLASSOPTIONcaptionsoff
%  \usepackage[nomarkers]{endfloat}
% \let\MYoriglatexcaption\caption
% \renewcommand{\caption}[2][\relax]{\MYoriglatexcaption[#2]{#2}}
%\fi
% endfloat.sty was written by James Darrell McCauley and Jeff Goldberg.
% This package may be useful when used in conjunction with IEEEtran.cls'
% captionsoff option. Some IEEE journals/societies require that submissions
% have lists of figures/tables at the end of the paper and that
% figures/tables without any captions are placed on a page by themselves at
% the end of the document. If needed, the draftcls IEEEtran class option or
% \CLASSINPUTbaselinestretch interface can be used to increase the line
% spacing as well. Be sure and use the nomarkers option of endfloat to
% prevent endfloat from "marking" where the figures would have been placed
% in the text. The two hack lines of code above are a slight modification of
% that suggested by in the endfloat docs (section 8.3.1) to ensure that
% the full captions always appear in the list of figures/tables - even if
% the user used the short optional argument of \caption[]{}.
% IEEE papers do not typically make use of \caption[]'s optional argument,
% so this should not be an issue. A similar trick can be used to disable
% captions of packages such as subfig.sty that lack options to turn off
% the subcaptions:
% For subfig.sty:
% \let\MYorigsubfloat\subfloat
% \renewcommand{\subfloat}[2][\relax]{\MYorigsubfloat[]{#2}}
% For subfigure.sty:
% \let\MYorigsubfigure\subfigure
% \renewcommand{\subfigure}[2][\relax]{\MYorigsubfigure[]{#2}}
% However, the above trick will not work if both optional arguments of
% the \subfloat/subfig command are used. Furthermore, there needs to be a
% description of each subfigure *somewhere* and endfloat does not add
% subfigure captions to its list of figures. Thus, the best approach is to
% avoid the use of subfigure captions (many IEEE journals avoid them anyway)
% and instead reference/explain all the subfigures within the main caption.
% The latest version of endfloat.sty and its documentation can obtained at:
% http://www.ctan.org/tex-archive/macros/latex/contrib/endfloat/
%
% The IEEEtran \ifCLASSOPTIONcaptionsoff conditional can also be used
% later in the document, say, to conditionally put the References on a
% page by themselves.





% *** PDF, URL AND HYPERLINK PACKAGES ***
%
\usepackage{url}
% url.sty was written by Donald Arseneau. It provides better support for
% handling and breaking URLs. url.sty is already installed on most LaTeX
% systems. The latest version can be obtained at:
% http://www.ctan.org/tex-archive/macros/latex/contrib/misc/
% Read the url.sty source comments for usage information. Basically,
% \url{my_url_here}.





% *** Do not adjust lengths that control margins, column widths, etc. ***
% *** Do not use packages that alter fonts (such as pslatex).         ***
% There should be no need to do such things with IEEEtran.cls V1.6 and later.
% (Unless specifically asked to do so by the journal or conference you plan
% to submit to, of course. )


% correct bad hyphenation here
\hyphenation{op-tical net-works semi-conduc-tor}


\begin{document}
%
% paper title
% can use linebreaks \\ within to get better formatting as desired
\title{Predicting Song Popularity}


\author{Sam~Xu~\IEEEmembership{samx@stanford.edu},
        Joyce~Xu~\IEEEmembership{jexu@stanford.edu},
        and~Eric~Tang~\IEEEmembership{etang21@stanford.edu}}


% The paper headers
\markboth{CS 221 (Autumn 2018) Project Proposal}%
{}
% The only time the second header will appear is for the odd numbered pages
% after the title page when using the twoside option.


% make the title area
\maketitle


\begin{abstract}
This project aims to predict the popularity of songs by analysing acoustic features such as tempo, duration, mode, loudness, key; artist information such as location and popularity; and metadata such as release title and year. The codebase for the project is located at \texttt{\url{https://github.com/inSam/SoundScorer221}}
\end{abstract}



\section{Introduction}
The music industry is a multi-billion dollar industry which produces over a million songs each year in the United States alone. With such a large volume of music produced, identifying which songs will become popular is a highly lucrative business. We would like to predict the popularity of a song by analyzing its audio features and metadata. Such a tool would be valuable for record labels, streaming services, and average consumers; it would also help the research community understand what acoustic features are currently popular with the public.

\section{Related Work}
Myra et al 2018 [1] analyzed over 500,000 songs to predict popularity (appearance on the Top 100 song list) from accoustic features and the "popstar status" of its artist. Using a random forest classifier, Myra et al. obtained a roughly 70\% classification accuracy without using the "popstar status" feature, and a roughly 80\% success rate with the popstar feature.

In a 2015 CS229 project, Pham et al. attempted to predict song popularity using support vector machines, multi-layered perceptrons, and logistical regression. Pham et al. utilized auditory features, such as key, tempo, and volume, as well as metadata like artist name and album name. Their metric of success was a "hotness" rating from The Echo Nest, an online aggregator of song popularity metrics. Their system performed moderately well with a Mean Squared Error of about 0.15, measured over a popularity index between 0 and 1 [2].

In a 2017 work, Yang et al. used convolutional neural networks (CNNs) to extract complex auditory features from songs and predict their resulting popularity. They found that features extracted using CNNs outperform features solely extracted using simple linear-regression techniques on audio files. \cite{cnn_prediction}

\section{Methodology}

\subsection{Dataset}
Our dataset comprises songs extracted from the Spotify API, which provides acoustic features, author information, and song metadata. More abstract features such as dancebility, energy, and song popularity were pulled from the Echo Nest, whose database is incororated into the Spotify API.

As preliminary data, we pulled 10,000 songs from the Spotify API, which contains a plethora of pre-extracted features that are also provided by the Spotify and the Echo Nest. These features include:
\begin{itemize}
	\item Acoustic Features: tempo, duration, loudness
	\item Metadata: artist name and information
  \item Audio: song preview
  \item Echo Nest Features: "danceability", "energy", etc.
\end{itemize}

\par The six Echo Nest features are derived from the audio file but use Echo Nest's proprietary algorithm.

\par Each song is also labeled with a popularity score between $1$ and $100$. This popularity score will serve as our ground truth label, and we will use the features above to run our preliminary baseline and oracle tests.


\subsection{Concrete Example of Input and Output}
Here you can see an an Input and Output example for the song \textit{Never Gonna Give You Up}:
\begin{align*}
    "TRAXLZU12903D05F94 "&: \{ \\
  &"features": \{\\
  &artist\_name: Rick Astley \\
  &loudness: -7.75 \\
  &tempo: 113.359 \\
  &duration: 211.69587 \\
  &mode: 1 \\
  &energy: 48.149 \\
  &previewurl: p.scdn.co/mp3/99 \\
  ...
        &\},\\
    \}
\end{align*}
The output is simply the Spotify popularity rating, a score between 1 and 100 (in this case, 86). Although a song's popularity fluctuates over time, we predict only the current popularity rating.

\subsection{Implementation}
We started by importing 10,000 songs from the public Spotify dataset and extracting features from the Spotify API.\cite{spotify_api} For our baseline experimentation, we trained a Linear Regression classifier on this Spotify data.


\subsection{Evaluation Metric}

We use Mean Squared Error as our evaluation matric.

$$ \frac{\sum_{i=1}^n(\text{Predicted}_i-\text{Actual}_i)^2}{n} $$



\section{Preliminary results}
\subsection{Baseline}
For our baseline, we trained a linear regression classifier on seven features already stored in the Spotify database: "acousticness", "danceability", "energy", "loudness", "speechiness", "tempo", and "valence". We used an 85/15 train/test split for our database of 10,000 songs, and our linear regression classifier achieved a mean squared error of $61.1$.

\subsection{Oracle}
For our oracle, we simply allowed our program to look at the correct answer and return the true value. As a result, our oracle obtains a loss of 0.

\section{Next Steps}

\subsection{Improve Features}
We will extract more features from the Spotify database such as song age, location, audio sections, and signature. We will also try to construct more advanced features from historical data such as musical trends. Using Spotify's provided links to song previews, we may be able to extract more advanced features from the audio recordings of each song. This step will involve a lot more feature tuning and experimentation.

\subsection{Improve Models}
For future models, we will try to tackle this problem by using ensemble methods such as gradient boosting regressors and random forests, support vector regressors, and neural networks to predict popularity. In particular, neural networks may be able to extract more meaningful features from the audio tracks of each song.

\section{Challenges}
One challenge is to consider the effects of time on the popularity on the song. We may be able to tackle this by simply adding the age of the song as a additional feature.

Another challenge is that there is plenty of noise that influences a song's popularity, beyond its mere auditory features. While we have assumed a $100\%$ success rate with our Oracle, this may be unrealistic to achieve with a classifier.


% references section

% can use a bibliography generated by BibTeX as a .bbl file
% BibTeX documentation can be easily obtained at:
% http://www.ctan.org/tex-archive/biblio/bibtex/contrib/doc/
% The IEEEtran BibTeX style support page is at:
% http://www.michaelshell.org/tex/ieeetran/bibtex/
%\bibliographystyle{IEEEtran}
% argument is your BibTeX string definitions and bibliography database(s)
%\bibliography{IEEEabrv,../bib/paper}
%
% <OR> manually copy in the resultant .bbl file
% set second argument of \begin to the number of references
% (used to reserve space for the reference number labels box)
\begin{thebibliography}{1}
\bibitem{marketing}
Myra Interiano, Kamyar Kazemi, Lijia Wang, Jienian Yang, Zhaoxia Yu, Natalia L. Komarova, \emph{Musical trends and predictability of success in contemporary songs in and out of the top charts} \hskip 1em plus 0.5em
minus 0.4em\relax Royal Society Open Sciences, 2006.
\bibitem{greenlight}
James Pham, Edrick Kyauk, Edwin Park, \emph{Predicting Song Popularity}
\hskip 1em plus 0.5em minus 0.4em\relax Stanford University CS 229, 2015.
\bibitem{spotify_api}
\texttt{\url{https://developer.spotify.com/documentation/web-api/}}
\bibitem{cnn_prediction}
Li-Chia Yang, Szu-Yu Chou, Jen-Yu Liu, Yi-Hsuan Yang, Yi-An Chen, \emph{REVISITING THE PROBLEM OF AUDIO-BASED HIT SONG PREDICTION USING
CONVOLUTIONAL NEURAL NETWORKS}
\hskip 1em plus 0.5em minus 0.4em\relax Academia Sinica, Taiwan, 2017.

\end{thebibliography}


% that's all folks
\end{document}
